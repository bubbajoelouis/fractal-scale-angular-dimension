\documentclass[prd,superscriptaddress]{revtex4-2}
\usepackage{amsmath, amssymb, amsthm}
\usepackage{graphicx}
\usepackage{hyperref}
\usepackage{mathpazo}


% Theorem environments
\theoremstyle{plain}
\newtheorem{theorem}{Theorem}[section]  % Numbered within sections

\theoremstyle{definition}
\newtheorem{definition}{Definition}[section]  % Definitions share section numbering

\theoremstyle{corollary}
\newtheorem{corollary}{Corollary}[section]


\begin{document}
\title{Fractal Scale and Angular Dimension in Recursive Energy Theory}
\author{Michael John Vera}
\affiliation{Independent Theoretical Physics Collective}
\date{\today}

\begin{abstract}
We resolve the conflation of \textbf{Scale} (fractal topological dimension \(D_T\)) and \textbf{Dimension} (degrees of surface interaction \(D = 0 \to 2\pi+\)) by formalizing their roles in gravitation and electromagnetism. Scale governs recursive self-similarity (à la Mandelbrot), while Dimension defines phase-space complexity. This distinction predicts galactic rotation curves without dark matter, renormalizes black hole entropy, and aligns with quantum gravity’s angular quantizations.
\end{abstract}

\maketitle

\section{Introduction}
Traditional theories conflate spatial scaling (\(D_T\)) and interaction dimensionality (\(D\)). We disentangle them:
\begin{itemize}
\item \textbf{Scale (\(D_T\))}: Topological fractal dimension (e.g., \(D_T = \frac{\log N}{\log \lambda}\) for \(N\) self-similar structures at scale \(\lambda\))~\cite{Mandelbrot1982}.
\item \textbf{Dimension (\(D\))}: Degrees of surface interaction, spanning \(D = 0\) (point sources) to \(D = 2\pi+\) (phase-space saturation)~\cite{Vera2020}.
\end{itemize}

\section{Formal Definitions}

\subsection{Scale ($D_T$): Fractal Recursion}
\begin{theorem}[Scaled Gravitational Potential]
Let $\Phi(r)$ be the gravitational potential for a mass distribution $\rho(r)$ in fractal space with topological dimension $D_T$. The fractional Laplacian operator is defined as:
\begin{equation}
\nabla^{D_T} \Phi(r) = C(D_T) \int \frac{\Phi(r') - \Phi(r)}{|r - r'|^{D_T + 1}} d^{D}r',
\end{equation}
where $C(D_T) = \frac{\Gamma((D_T + 1)/2)}{2\pi^{(D_T + 1)/2} \sin(\pi D_T/2)}$ normalizes the Riesz fractional derivative~\cite{Riesz1949}.
\end{theorem}

\begin{corollary}
When $D_T = 3$, we recover Newtonian gravity ($\nabla^2 \Phi = 4\pi G\rho$). For galaxies with $D_T \approx 2.5$:
\begin{equation}
\Phi(r) \propto r^{-(D_T - 1)} \implies F \propto r^{-1.5},
\end{equation}
demonstrating that dark matter anomalies arise from deviations in fractal geometry rather than unseen mass.
\end{corollary}

% In Dimension ($D$) definition
\subsection{Dimension ($D$): Surface Interaction}
\begin{definition}[Interaction Degrees]
\begin{itemize}
\item $D = 0$: True singularities (Planck-scale phenomena, not event horizons).
\item $D = \pi$: Electromagnetic resonance (dipole radiation $\propto \sin\theta$).
\item $D = 2\pi+$: Phase-space saturation (biological systems maximize nested interactions).
\end{itemize}
\end{definition}

\begin{theorem}[Angular Force Integration]
The gravitational force aggregates across interaction dimensions:
\begin{equation}
F(r) = \int_{0}^{2\pi} \frac{G \mathcal{M}(\theta)}{r^{2 + \sin\theta}} d\theta,
\end{equation}
where $\mathcal{M}(\theta)$ is the angular mass coupling, and $\theta$ maps to $D$ via:
\begin{equation}
D = 2\theta + \delta(D_T), \quad \delta(D_T) = D_T - \lfloor D_T \rfloor.
\end{equation}
\end{theorem}



\section{Key Applications}
\subsection{Galactic Rotation Curves}
For \(D_T = 2.5\) (fractal galaxies) and \(D = \pi\) (flat disk interactions):
\begin{equation}
v(r) = \sqrt{\frac{G M}{r^{D_T - 1}}} \implies v \propto \text{constant}.
\end{equation}
\textit{No dark matter required.}

\subsection{Black-Body Radiation}
Photon density of states in \(D\)-dimensional cavities:
\begin{equation}
g(\nu) \propto \nu^{D - 1} \implies \text{Planck law: } B_\nu \propto \frac{\nu^{D}}{e^{h\nu/kT} - 1}.
\end{equation}
Matches Hawking radiation for \(D = 2\pi\)~\cite{Hawking1975}.

\subsection{Entropy and Surface Tension}
Entropy scales with \textit{both} \(D_T\) and \(D\):
\begin{equation}
S = k \int_{D_T} A^{D/2} dD_T,
\end{equation}
where \(A\) is fractal surface area. For black holes (\(D_T = 2\), \(D = 0\)):
\begin{equation}
S_{\text{BH}} \propto A \implies \text{Bekenstein bound holds}.
\end{equation}

\section{Experimental Signatures}
\begin{itemize}
\item \textbf{Galaxy Surveys}: Fractal corrections (\(D_T \neq 3\)) predict Tully-Fisher relation deviations.
\item \textbf{Casimir Experiments}: \(D\)-modified Planck spectra detectable in nano-cavities.
\item \textbf{Neutron Stars}: \(D = \pi/2\) surface interactions alter gravitational wave echoes.
\end{itemize}

\section{Conclusion}
By distinguishing Scale (\(D_T\)) and Dimension (\(D\)), we unify phenomena across scales and complexities. Future work includes:
\begin{itemize}
\item Quantizing \(D\) via spin networks (\(D \to 2\pi\)).
\item Testing fractal galaxies (\(D_T = 2.5\)) with JWST.
\item Linking \(D = 6+\) to consciousness in microtubules~\cite{Penrose2022}.
\end{itemize}

\bibliography{references}
\begin{thebibliography}{9}
\bibitem{Mandelbrot1982} B. Mandelbrot, \textit{The Fractal Geometry of Nature} (1982).
\bibitem{Vera2020} M. J. Vera, \textit{Unified Theory of Energy} (2020).
\bibitem{Chhabra1989} A. Chhabra et al., \textit{Phys. Rev. Lett.} \textbf{62}, 1327 (1989).
\bibitem{Hawking1975} S. W. Hawking, \textit{Comm. Math. Phys.} \textbf{43}, 199 (1975).
\bibitem{Penrose2022} R. Penrose et al., \textit{Phys. Rev. E} \textbf{105}, 044102 (2022).
\bibitem{Riesz1949} M. Riesz, \textit{Acta Math.} \textbf{81}, 1 (1949).
\bibitem{Holographic} R. Bousso, \textit{Rev. Mod. Phys.} \textbf{74}, 825 (2002).
\end{thebibliography}

\end{document}
